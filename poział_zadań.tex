\documentclass[a4paper,12pt]{article}

\usepackage[utf8]{inputenc}
\usepackage[T1]{fontenc}
\usepackage[polish]{babel}
\usepackage[margin=2.5cm]{geometry}
\usepackage{parskip}
\usepackage{xcolor}
\usepackage{enumitem}
\usepackage{titlesec}
\definecolor{pyblue}{RGB}{48, 105, 152}
\definecolor{cppred}{RGB}{150, 0, 20}

\titleformat{\section}
{\normalfont\Large\bfseries}{\thesection}{1em}{}

\title{\textbf{Plan Projektu Zespołowego: Solver Kostki Rubika}}
\author{Zespół Projektowy (6 osób)}
\date{\today}

\begin{document}

\maketitle

\section*{Wstęp}
Celem projektu jest stworzenie aplikacji rozwiązującej Kostkę Rubika, składającej się z interfejsu graficznego (Python) oraz silnika obliczeniowego (C++). Weryfikacja rozwiązywalności kostki opiera się na limicie czasowym (timeout) – jeśli silnik C++ nie znajdzie rozwiązania w zadanym czasie, stan uznawany jest za błędny.

\hrule
\vspace{0.5cm}

\section{\textcolor{pyblue}{Zespół Python (Warstwa Prezentacji i Sterowania)}}

\subsection*{Osoba P1: Graficzne Wprowadzanie Danych (Input GUI)}
\textbf{Główne zadanie:} Stworzenie interaktywnej planszy do definiowania stanu początkowego kostki.
\begin{itemize}[label=$\bullet$]
    \item Implementacja interfejsu (np. Tkinter, PyGame, PyQt) z siatką kostki.
    \item Obsługa wyboru kolorów ("paleta") i kolorowania pól przez użytkownika.
    \item Zapisanie zdefiniowanego stanu do pliku tekstowego (np. \texttt{input.txt}) w ustalonym formacie.
    \item Podstawowa weryfikacja (np. czy nie pozostawiono pustych pól).
\end{itemize}

\subsection*{Osoba P2: Graficzna Wizualizacja Rozwiązania (Output GUI)}
\textbf{Główne zadanie:} Prezentacja instrukcji ułożenia kostki na podstawie wyników silnika.
\begin{itemize}[label=$\bullet$]
    \item Odczyt pliku wynikowego z C++ (np. \texttt{output.txt}) zawierającego listę ruchów.
    \item Implementacja wizualizacji 3D lub animowanej siatki 2D.
    \item Obsługa przycisków sterujących: ,,Start'', ,,Następny ruch'', ,,Poprzedni ruch''.
\end{itemize}

\subsection*{Osoba P3: Terminal, Orkiestracja i Dokumentacja (CLI \& Integration)}
\textbf{Główne zadanie:} Zarządzanie procesem C++, obsługa terminala i spójność repozytorium.
\begin{itemize}[label=$\bullet$]
    \item \textbf{Implementacja mechanizmu Timeout:} Uruchamianie silnika C++ jako podprocesu (\texttt{subprocess}). Przechwytywanie błędu, jeśli rozwiązanie trwa zbyt długo (oznacza to kostkę nierozwiązywalną).
    \item Obsługa trybu tekstowego (CLI): wczytywanie stanu i wypisywanie kroków w terminalu.
    \item Zarządzanie dokumentacją (README), strukturą plików i scalaniem kodu (Merge Master).
\end{itemize}

\vspace{0.5cm}

\section{\textcolor{cppred}{Zespół C++ (Silnik Obliczeniowy)}}

\subsection*{Osoba C1: Struktura Danych i Ruchy (Core \& I/O)}
\textbf{Główne zadanie:} Fundament silnika – reprezentacja kostki i szybkość operacji.
\begin{itemize}[label=$\bullet$]
    \item Projekt klasy \texttt{Cube} (reprezentacja ścianek w pamięci).
    \item Implementacja mechaniki obrotów (U, D, L, R, F, B).
    \item Obsługa wejścia/wyjścia: parsowanie \texttt{input.txt} i formatowanie \texttt{output.txt}.
    \item Optymalizacja wydajności podstawowych operacji na kostce.
\end{itemize}

\subsection*{Osoba C2: Algorytm Przeszukiwania (Solver)}
\textbf{Główne zadanie:} Logika znajdująca sekwencję rozwiązującą.
\begin{itemize}[label=$\bullet$]
    \item Implementacja głównej pętli algorytmu (np. IDA*, Kociemba).
    \item Zarządzanie stanami odwiedzonymi i kolejką priorytetową.
    \item Współpraca z modułem heurystyk (C3) w celu kierowania przeszukiwaniem.
    \item Ciągła praca algorytmu aż do znalezienia rozwiązania lub przerwania przez proces nadrzędny (Python).
\end{itemize}

\subsection*{Osoba C3: Heurystyka i Optymalizacja (Brain)}
\textbf{Główne zadanie:} Przyspieszenie algorytmu, aby unikać Timeoutu przy poprawnych kostkach.
\begin{itemize}[label=$\bullet$]
    \item Implementacja funkcji heurystycznej (ocena odległości od celu).
    \item Generowanie i obsługa tablic przejść/przycięć (Pattern Databases / Pruning Tables).
    \item Optymalizacja zużycia pamięci przez algorytm przeszukiwania.
\end{itemize}

\end{document}
